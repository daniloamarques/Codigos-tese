\documentclass[a4paper,12pt]{article}

% --- Pacotes essenciais ---
\usepackage[utf8]{inputenc}   % Codificação de caracteres
\usepackage[T1]{fontenc}      % Acentos corretos
\usepackage[brazil]{babel}    % Idioma português do Brasil
\usepackage{booktabs}         % Linhas mais elegantes nas tabelas
\usepackage{caption}          % Melhor controle sobre legendas
\usepackage{geometry}         % Controle de margens
\usepackage{array}            % Controle de largura das colunas
\usepackage{multirow}         % Combinar células verticalmente
\usepackage{float}            % Permite usar [H] para fixar tabelas
\usepackage{siunitx}          % Alinhamento numérico e formatação de números
\usepackage{pgfplots}
\pgfplotsset{compat=1.18}
% --- Configurações de layout ---
\geometry{a4paper, margin=2.5cm}
\captionsetup{justification=centering, font=small}

% --- Formatação de números decimais (opcional, vírgula como separador) ---
\sisetup{
  output-decimal-marker = {,}
}

\begin{document}


\begin{table}[h!]
\centering
\caption{Número de pontos umbílicos em Homens}
\begin{tabular}{c c}
\hline
\textbf{Número de umbílicos} & \textbf{Quantidade de indivíduos} \\ 
\hline
5  & 3   \\
6  & 8   \\
7  & 33  \\
8  & 56  \\
9  & 111 \\
10 & 130 \\
11 & 126 \\
12 & 70  \\
13 & 36  \\
14 & 9   \\
15 & 3   \\
16 & 1   \\
\hline
\end{tabular}
\end{table}

\begin{figure}[h!]
\centering
\begin{tikzpicture}
\begin{axis}[
    ybar,
    bar width=8pt,
    xlabel={Número de umbílicos},
    ylabel={Quantidade de indivíduos},
    xtick=data,
    enlarge x limits=0.05,
    ymin=0,
    width=14cm,
    height=8cm,
    grid=major,
]
\addplot coordinates {
    (5,3)
    (6,8)
    (7,33)
    (8,56)
    (9,111)
    (10,130)
    (11,126)
    (12,70)
    (13,36)
    (14,9)
    (15,3)
    (16,1)
};
\end{axis}
\end{tikzpicture}
\caption{Número de pontos umbílicos em Homens}
\end{figure}

\begin{table}[h!]
\centering
\caption{Região dos pontos umbílicos em Homens}
\begin{tabular}{l c c}
\hline
\textbf{Região} & \textbf{Número de umbílicos} & \textbf{Quantidade de indivíduos} \\
\hline
Nariz & 1 & 26 \\
      & 2 & 388 \\
      & 3 & 167 \\
      & 4 & 5 \\
\hline
Boca  & 0 & 392 \\
      & 1 & 145 \\
      & 2 & 44 \\
      & 3 & 5 \\
\hline
Testa & 0 & 9 \\
      & 1 & 140 \\
      & 2 & 227 \\
      & 3 & 167 \\
      & 4 & 41 \\
      & 5 & 2 \\
\hline
Olho Direito & 0 & 586 \\
\hline
Bochecha Direita & 0 & 42 \\
                 & 1 & 321 \\
                 & 2 & 216 \\
                 & 3 & 7 \\
\hline
Queixo & 1 & 51 \\
       & 2 & 138 \\
       & 3 & 266 \\
       & 4 & 125 \\
       & 5 & 6 \\
\hline
Buço & 0 & 586 \\
\hline
Olho Esquerdo & 0 & 584 \\
              & 1 & 2 \\
\hline
Bochecha Esquerda & 0 & 114 \\
                  & 1 & 306 \\
                  & 2 & 153 \\
                  & 3 & 13 \\
\hline
\end{tabular}
\end{table}


\begin{table}[h!]
\centering
\caption{Número de pontos umbílicos em Mulheres}
\begin{tabular}{c c}
\hline
\textbf{Número de umbílicos} & \textbf{Quantidade de indivíduos} \\ 
\hline
5  & 4   \\
6  & 6   \\
7  & 31  \\
8  & 53  \\
9  & 77  \\
10 & 79  \\
11 & 68  \\
12 & 48  \\
13 & 19  \\
14 & 8   \\
15 & 6   \\
16 & 2   \\
\hline
\end{tabular}
\end{table}

\begin{figure}[h!]
\centering
\begin{tikzpicture}
\begin{axis}[
    ybar,
    bar width=8pt,
    xlabel={Número de umbílicos},
    ylabel={Quantidade de indivíduos},
    xtick=data,
    enlarge x limits=0.05,
    ymin=0,
    width=14cm,
    height=8cm,
    grid=major,
]
\addplot coordinates {
    (5,4)
    (6,6)
    (7,31)
    (8,53)
    (9,77)
    (10,79)
    (11,68)
    (12,48)
    (13,19)
    (14,8)
    (15,6)
    (16,2)
};
\end{axis}
\end{tikzpicture}
\caption{Número de pontos umbílicos em Mulheres}
\end{figure}

\begin{table}[h!]
\centering
\caption{Região dos pontos umbílicos em Mulheres}
\begin{tabular}{l c c}
\hline
\textbf{Região} & \textbf{Número de umbílicos} & \textbf{Quantidade de indivíduos} \\
\hline
Nariz & 1 & 12 \\
      & 2 & 293 \\
      & 3 & 93 \\
      & 4 & 3 \\
\hline
Boca  & 0 & 307 \\
      & 1 & 80 \\
      & 2 & 13 \\
      & 3 & 1 \\
\hline
Testa & 0 & 4 \\
      & 1 & 66 \\
      & 2 & 186 \\
      & 3 & 113 \\
      & 4 & 30 \\
      & 5 & 1 \\
      & 6 & 1 \\
\hline
Olho Direito & 0 & 401 \\
\hline
Bochecha Direita & 0 & 27 \\
                 & 1 & 227 \\
                 & 2 & 138 \\
                 & 3 & 9 \\
\hline
Queixo & 0 & 5 \\
       & 1 & 22 \\
       & 2 & 101 \\
       & 3 & 168 \\
       & 4 & 95 \\
       & 5 & 10 \\
\hline
Buço & 0 & 401 \\
\hline
Olho Esquerdo & 0 & 401 \\
\hline
Bochecha Esquerda & 0 & 106 \\
                  & 1 & 201 \\
                  & 2 & 93 \\
                  & 3 & 1 \\
\hline
\end{tabular}
\end{table}




\begin{table}[h!]
\centering
\caption{Número de pontos umbílicos — Ambos os sexos}
\begin{tabular}{c c}
\hline
\textbf{Número de umbílicos} & \textbf{Quantidade de indivíduos} \\ 
\hline
5  & 7   \\
6  & 14  \\
7  & 64  \\
8  & 109 \\
9  & 188 \\
10 & 209 \\
11 & 194 \\
12 & 118 \\
13 & 55  \\
14 & 17  \\
15 & 9   \\
16 & 3   \\
\hline
\end{tabular}
\end{table}

\begin{figure}[h!]
\centering
\begin{tikzpicture}
\begin{axis}[
    ybar,
    bar width=8pt,
    xlabel={Número de umbílicos},
    ylabel={Quantidade de indivíduos},
    xtick=data,
    enlarge x limits=0.05,
    ymin=0,
    width=14cm,
    height=8cm,
    grid=major,
]
\addplot coordinates {
    (5,7)
    (6,14)
    (7,64)
    (8,109)
    (9,188)
    (10,209)
    (11,194)
    (12,118)
    (13,55)
    (14,17)
    (15,9)
    (16,3)
};
\end{axis}
\end{tikzpicture}
\caption{Número de pontos umbílicos — Ambos os sexos}
\end{figure}

\begin{table}[h!]
\centering
\caption{Região dos pontos umbílicos — Ambos os sexos}
\begin{tabular}{l c c}
\hline
\textbf{Região} & \textbf{Número de umbílicos} & \textbf{Quantidade de indivíduos} \\
\hline
Nariz & 1 & 38 \\
      & 2 & 681 \\
      & 3 & 260 \\
      & 4 & 8 \\
\hline
Boca  & 0 & 699 \\
      & 1 & 225 \\
      & 2 & 57 \\
      & 3 & 6 \\
\hline
Testa & 0 & 13 \\
      & 1 & 206 \\
      & 2 & 413 \\
      & 3 & 280 \\
      & 4 & 71 \\
      & 5 & 3 \\
      & 6 & 1 \\
\hline
Olho Direito & 0 & 987 \\
\hline
Bochecha Direita & 0 & 69 \\
                 & 1 & 548 \\
                 & 2 & 354 \\
                 & 3 & 16 \\
\hline
Queixo & 0 & 5 \\
       & 1 & 73 \\
       & 2 & 239 \\
       & 3 & 434 \\
       & 4 & 220 \\
       & 5 & 16 \\
\hline
Buço & 0 & 987 \\
\hline
Olho Esquerdo & 0 & 985 \\
              & 1 & 2 \\
\hline
Bochecha Esquerda & 0 & 220 \\
                  & 1 & 507 \\
                  & 2 & 246 \\
                  & 3 & 14 \\
\hline
\end{tabular}
\end{table}





\end{document}