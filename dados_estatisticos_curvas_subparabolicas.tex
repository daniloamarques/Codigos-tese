\documentclass[a4paper,12pt]{article}

% --- Pacotes essenciais ---
\usepackage[utf8]{inputenc}   % Codificação de caracteres
\usepackage[T1]{fontenc}      % Acentos corretos
\usepackage[brazil]{babel}    % Idioma português do Brasil
\usepackage{booktabs}         % Linhas mais elegantes nas tabelas
\usepackage{caption}          % Melhor controle sobre legendas
\usepackage{geometry}         % Controle de margens
\usepackage{array}            % Controle de largura das colunas
\usepackage{multirow}         % Combinar células verticalmente
\usepackage{float}            % Permite usar [H] para fixar tabelas
\usepackage{siunitx}          % Alinhamento numérico e formatação de números

% --- Configurações de layout ---
\geometry{a4paper, margin=2.5cm}
\captionsetup{justification=centering, font=small}

% --- Formatação de números decimais opcional, vírgula como separador ---
\sisetup{
  output-decimal-marker = {,}
}

\begin{document}

\begin{table}[h!]
\centering
\caption{Número de curvas subparabólicas azul — Homens}
\begin{tabular}{c c}
\hline
\textbf{Número de curvas subparabólicas azul} & \textbf{Quantidade de indivíduos} \\ 
\hline
14 & 2  \\
16 & 6  \\
17 & 12 \\
18 & 27 \\
19 & 30 \\
20 & 41 \\
21 & 53 \\
22 & 56 \\
23 & 80 \\
24 & 74 \\
25 & 52 \\
26 & 62 \\
27 & 27 \\
28 & 23 \\
29 & 22 \\
30 & 10 \\
31 & 4  \\
32 & 3  \\
33 & 1  \\
35 & 1  \\
\hline
\end{tabular}
\end{table}


\begin{table}[h!]
\centering
\caption{Número de curvas subparabólicas azul — Mulheres}
\begin{tabular}{c c}
\hline
\textbf{Número de curvas subparabólicas azul} & \textbf{Quantidade de indivíduos} \\ 
\hline
14 & 1  \\
15 & 9  \\
16 & 4  \\
17 & 12 \\
18 & 24 \\
19 & 33 \\
20 & 39 \\
21 & 47 \\
22 & 52 \\
23 & 37 \\
24 & 50 \\
25 & 36 \\
26 & 25 \\
27 & 14 \\
28 & 8  \\
29 & 4  \\
30 & 3  \\
32 & 2  \\
34 & 1  \\
\hline
\end{tabular}
\end{table}


\begin{table}[h!]
\centering
\caption{Número de curvas subparabólicas azul — Ambos os sexos}
\begin{tabular}{c c}
\hline
\textbf{Número de curvas subparabólicas azul} & \textbf{Quantidade de indivíduos} \\ 
\hline
14 & 3  \\
15 & 9  \\
16 & 10 \\
17 & 24 \\
18 & 51 \\
19 & 63 \\
20 & 80 \\
21 & 100 \\
22 & 108 \\
23 & 117 \\
24 & 124 \\
25 & 88 \\
26 & 87 \\
27 & 41 \\
28 & 31 \\
29 & 26 \\
30 & 13 \\
31 & 4  \\
32 & 5  \\
33 & 1  \\
34 & 1  \\
35 & 1  \\
\hline
\end{tabular}
\end{table}


\begin{table}[h!]
\centering
\caption{Número de curvas subparabólicas vermelha — Homens}
\begin{tabular}{c c}
\hline
\textbf{Número de curvas subparabólicas vermelha} & \textbf{Quantidade de indivíduos} \\ 
\hline
10 & 1  \\
12 & 1  \\
13 & 1  \\
14 & 10 \\
15 & 19 \\
16 & 22 \\
17 & 49 \\
18 & 60 \\
19 & 61 \\
20 & 65 \\
21 & 69 \\
22 & 66 \\
23 & 57 \\
24 & 39 \\
25 & 27 \\
26 & 21 \\
27 & 9  \\
28 & 5  \\
30 & 4  \\
\hline
\end{tabular}
\end{table}


\begin{table}[h!]
\centering
\caption{Número de curvas subparabólicas vermelha — Mulheres}
\begin{tabular}{c c}
\hline
\textbf{Número de curvas subparabólicas vermelha} & \textbf{Quantidade de indivíduos} \\ 
\hline
11 & 1  \\
14 & 3  \\
15 & 7  \\
16 & 7  \\
17 & 21 \\
18 & 37 \\
19 & 40 \\
20 & 58 \\
21 & 39 \\
22 & 47 \\
23 & 43 \\
24 & 36 \\
25 & 32 \\
26 & 14 \\
27 & 8  \\
28 & 3  \\
29 & 3  \\
30 & 2  \\
\hline
\end{tabular}
\end{table}


\begin{table}[h!]
\centering
\caption{Número de curvas subparabólicas vermelha — Ambos os sexos}
\begin{tabular}{c c}
\hline
\textbf{Número de curvas subparabólicas vermelha} & \textbf{Quantidade de indivíduos} \\ 
\hline
10 & 1  \\
11 & 1  \\
12 & 1  \\
13 & 1  \\
14 & 13 \\
15 & 26 \\
16 & 29 \\
17 & 70 \\
18 & 97 \\
19 & 101 \\
20 & 123 \\
21 & 108 \\
22 & 113 \\
23 & 100 \\
24 & 75 \\
25 & 59 \\
26 & 35 \\
27 & 17 \\
28 & 8  \\
29 & 3  \\
30 & 6  \\
\hline
\end{tabular}
\end{table}


\begin{table}[h!]
\centering
\caption{Número de curvas subparabólicas azul fechada — Homens}
\begin{tabular}{c c}
\hline
\textbf{Número de curvas subparabólicas azul fechada} & \textbf{Quantidade de indivíduos} \\ 
\hline
4  & 1  \\
5  & 1  \\
6  & 8  \\
7  & 20 \\
8  & 35 \\
9  & 45 \\
10 & 63 \\
11 & 85 \\
12 & 73 \\
13 & 65 \\
14 & 60 \\
15 & 45 \\
16 & 29 \\
17 & 35 \\
18 & 12 \\
19 & 6  \\
20 & 1  \\
23 & 1  \\
24 & 1  \\
\hline
\end{tabular}
\end{table}


\begin{table}[h!]
\centering
\caption{Número de curvas subparabólicas azul fechada — Mulheres}
\begin{tabular}{c c}
\hline
\textbf{Número de curvas subparabólicas azul fechada} & \textbf{Quantidade de indivíduos} \\ 
\hline
4  & 2  \\
5  & 4  \\
6  & 6  \\
7  & 18 \\
8  & 33 \\
9  & 40 \\
10 & 57 \\
11 & 48 \\
12 & 61 \\
13 & 43 \\
14 & 37 \\
15 & 24 \\
16 & 15 \\
17 & 4  \\
18 & 3  \\
19 & 4  \\
20 & 1  \\
21 & 1  \\
\hline
\end{tabular}
\end{table}


\begin{table}[h!]
\centering
\caption{Número de curvas subparabólicas azul fechada — Ambos os sexos}
\begin{tabular}{c c}
\hline
\textbf{Número de curvas subparabólicas azul fechada} & \textbf{Quantidade de indivíduos} \\ 
\hline
4  & 3  \\
5  & 5  \\
6  & 14 \\
7  & 38 \\
8  & 68 \\
9  & 85 \\
10 & 120 \\
11 & 133 \\
12 & 134 \\
13 & 108 \\
14 & 97 \\
15 & 69 \\
16 & 44 \\
17 & 39 \\
18 & 15 \\
19 & 10 \\
20 & 2  \\
21 & 1  \\
23 & 1  \\
24 & 1  \\
\hline
\end{tabular}
\end{table}


\begin{table}[h!]
\centering
\caption{Número de curvas subparabólicas azul aberta — Homens}
\begin{tabular}{c c}
\hline
\textbf{Número de curvas subparabólicas azul aberta} & \textbf{Quantidade de indivíduos} \\ 
\hline
5  & 2  \\
6  & 7  \\
7  & 17 \\
8  & 38 \\
9  & 81 \\
10 & 88 \\
11 & 92 \\
12 & 103 \\
13 & 69 \\
14 & 46 \\
15 & 28 \\
16 & 12 \\
17 & 2  \\
18 & 1  \\
\hline
\end{tabular}
\end{table}


\begin{table}[h!]
\centering
\caption{Número de curvas subparabólicas azul aberta — Mulheres}
\begin{tabular}{c c}
\hline
\textbf{Número de curvas subparabólicas azul aberta} & \textbf{Quantidade de indivíduos} \\ 
\hline
5  & 1  \\
6  & 3  \\
7  & 17 \\
8  & 36 \\
9  & 49 \\
10 & 82 \\
11 & 73 \\
12 & 50 \\
13 & 47 \\
14 & 26 \\
15 & 9  \\
16 & 6  \\
17 & 2  \\
\hline
\end{tabular}
\end{table}


\begin{table}[h!]
\centering
\caption{Número de curvas subparabólicas azul aberta — Ambos os sexos}
\begin{tabular}{c c}
\hline
\textbf{Número de curvas subparabólicas azul aberta} & \textbf{Quantidade de indivíduos} \\ 
\hline
5  & 3  \\
6  & 10 \\
7  & 34 \\
8  & 74 \\
9  & 130 \\
10 & 170 \\
11 & 165 \\
12 & 153 \\
13 & 116 \\
14 & 72 \\
15 & 37 \\
16 & 18 \\
17 & 4  \\
18 & 1  \\
\hline
\end{tabular}
\end{table}


\begin{table}[h!]
\centering
\caption{Número de curvas subparabólicas vermelha fechada — Homens}
\begin{tabular}{c c}
\hline
\textbf{Número de curvas subparabólicas vermelha fechada} & \textbf{Quantidade de indivíduos} \\ 
\hline
3  & 1  \\
4  & 2  \\
5  & 8  \\
6  & 9  \\
7  & 23 \\
8  & 43 \\
9  & 44 \\
10 & 69 \\
11 & 74 \\
12 & 75 \\
13 & 70 \\
14 & 59 \\
15 & 39 \\
16 & 26 \\
17 & 20 \\
18 & 14 \\
19 & 6  \\
20 & 2  \\
21 & 1  \\
22 & 1  \\
\hline
\end{tabular}
\end{table}


\begin{table}[h!]
\centering
\caption{Número de curvas subparabólicas vermelha fechada — Mulheres}
\begin{tabular}{c c}
\hline
\textbf{Número de curvas subparabólicas vermelha fechada} & \textbf{Quantidade de indivíduos} \\ 
\hline
2  & 1  \\
5  & 1  \\
6  & 3  \\
7  & 10 \\
8  & 17 \\
9  & 31 \\
10 & 37 \\
11 & 51 \\
12 & 52 \\
13 & 37 \\
14 & 52 \\
15 & 40 \\
16 & 26 \\
17 & 23 \\
18 & 12 \\
19 & 4  \\
20 & 3  \\
22 & 1  \\
\hline
\end{tabular}
\end{table}


\begin{table}[h!]
\centering
\caption{Número de curvas subparabólicas vermelha fechada — Ambos os sexos}
\begin{tabular}{c c}
\hline
\textbf{Número de curvas subparabólicas vermelha fechada} & \textbf{Quantidade de indivíduos} \\ 
\hline
2  & 1  \\
3  & 1  \\
4  & 2  \\
5  & 9  \\
6  & 12 \\
7  & 33 \\
8  & 60 \\
9  & 75 \\
10 & 106 \\
11 & 125 \\
12 & 127 \\
13 & 107 \\
14 & 111 \\
15 & 79 \\
16 & 52 \\
17 & 43 \\
18 & 26 \\
19 & 10 \\
20 & 5  \\
21 & 1  \\
22 & 2  \\
\hline
\end{tabular}
\end{table}


\begin{table}[h!]
\centering
\caption{Número de curvas subparabólicas vermelha aberta — Homens}
\begin{tabular}{c c}
\hline
\textbf{Número de curvas subparabólicas vermelha aberta} & \textbf{Quantidade de indivíduos} \\ 
\hline
4  & 3  \\
5  & 12 \\
6  & 41 \\
7  & 95 \\
8  & 115 \\
9  & 129 \\
10 & 92 \\
11 & 65 \\
12 & 24 \\
13 & 8  \\
14 & 1  \\
15 & 1  \\
\hline
\end{tabular}
\end{table}


\begin{table}[h!]
\centering
\caption{Número de curvas subparabólicas vermelha aberta — Mulheres}
\begin{tabular}{c c}
\hline
\textbf{Número de curvas subparabólicas vermelha aberta} & \textbf{Quantidade de indivíduos} \\ 
\hline
4  & 1  \\
5  & 6  \\
6  & 27 \\
7  & 72 \\
8  & 100\\
9  & 78 \\
10 & 52 \\
11 & 36 \\
12 & 20 \\
13 & 9  \\
\hline
\end{tabular}
\end{table}


\begin{table}[h!]
\centering
\caption{Número de curvas subparabólicas vermelha aberta — Ambos os sexos}
\begin{tabular}{c c}
\hline
\textbf{Número de curvas subparabólicas vermelha aberta} & \textbf{Quantidade de indivíduos} \\ 
\hline
4  & 4   \\
5  & 18  \\
6  & 68  \\
7  & 167 \\
8  & 215 \\
9  & 207 \\
10 & 144 \\
11 & 101 \\
12 & 44  \\
13 & 17  \\
14 & 1   \\
15 & 1   \\
\hline
\end{tabular}
\end{table}







\end{document}