\documentclass[a4paper,12pt]{article}

% --- Pacotes essenciais ---
\usepackage[utf8]{inputenc}   % Codificação de caracteres
\usepackage[T1]{fontenc}      % Acentos corretos
\usepackage[brazil]{babel}    % Idioma português do Brasil
\usepackage{booktabs}         % Linhas mais elegantes nas tabelas
\usepackage{caption}          % Melhor controle sobre legendas
\usepackage{geometry}         % Controle de margens
\usepackage{array}            % Controle de largura das colunas
\usepackage{multirow}         % Combinar células verticalmente
\usepackage{float}            % Permite usar [H] para fixar tabelas
\usepackage{siunitx}          % Alinhamento numérico e formatação de números
\usepackage{pgfplots}
\pgfplotsset{compat=1.18}
% --- Configurações de layout ---
\geometry{a4paper, margin=2.5cm}
\captionsetup{justification=centering, font=small}

% --- Formatação de números decimais opcional, vírgula como separador ---
\sisetup{
  output-decimal-marker = {,}
}

\begin{document}

\begin{table}[h!]
\centering
\caption{Número de curvas ridges azul — Homens}
\begin{tabular}{c c}
\hline
\textbf{Número de curvas ridges azul} & \textbf{Quantidade de indivíduos} \\ 
\hline
15 & 1  \\
16 & 5  \\
17 & 11 \\
18 & 15 \\
19 & 26 \\
20 & 38 \\
21 & 56 \\
22 & 67 \\
23 & 61 \\
24 & 81 \\
25 & 67 \\
26 & 46 \\
27 & 47 \\
28 & 30 \\
29 & 17 \\
30 & 9  \\
31 & 4  \\
32 & 3  \\
33 & 2  \\
\hline
\end{tabular}
\end{table}

\begin{figure}[h!]
\centering
\begin{tikzpicture}
\begin{axis}[
    ybar,
    bar width=7pt,
    xlabel={Número de curvas ridges azul},
    ylabel={Quantidade de indivíduos},
    xtick=data,
    enlarge x limits=0.05,
    ymin=0,
    width=15cm,
    height=9cm,
    grid=major,
]
\addplot coordinates {
    (15,1)
    (16,5)
    (17,11)
    (18,15)
    (19,26)
    (20,38)
    (21,56)
    (22,67)
    (23,61)
    (24,81)
    (25,67)
    (26,46)
    (27,47)
    (28,30)
    (29,17)
    (30,9)
    (31,4)
    (32,3)
    (33,2)
};
\end{axis}
\end{tikzpicture}
\caption{Número de curvas ridges azul — Homens}
\end{figure}


\begin{table}[h!]
\centering
\caption{Número de curvas ridges azul — Mulheres}
\begin{tabular}{c c}
\hline
\textbf{Número de curvas ridges azul} & \textbf{Quantidade de indivíduos} \\ 
\hline
14 & 1  \\
16 & 2  \\
17 & 2  \\
18 & 8  \\
19 & 12 \\
20 & 21 \\
21 & 39 \\
22 & 42 \\
23 & 44 \\
24 & 61 \\
25 & 41 \\
26 & 43 \\
27 & 24 \\
28 & 25 \\
29 & 19 \\
30 & 5  \\
31 & 6  \\
32 & 1  \\
33 & 1  \\
34 & 1  \\
35 & 2  \\
38 & 1  \\
\hline
\end{tabular}
\end{table}

\begin{figure}[h!]
\centering
\begin{tikzpicture}
\begin{axis}[
    ybar,
    bar width=7pt,
    xlabel={Número de curvas ridges azul},
    ylabel={Quantidade de indivíduos},
    xtick=data,
    enlarge x limits=0.05,
    ymin=0,
    width=15cm,
    height=9cm,
    grid=major,
]
\addplot coordinates {
    (14,1)
    (16,2)
    (17,2)
    (18,8)
    (19,12)
    (20,21)
    (21,39)
    (22,42)
    (23,44)
    (24,61)
    (25,41)
    (26,43)
    (27,24)
    (28,25)
    (29,19)
    (30,5)
    (31,6)
    (32,1)
    (33,1)
    (34,1)
    (35,2)
    (38,1)
};
\end{axis}
\end{tikzpicture}
\caption{Número de curvas ridges azul — Mulheres}
\end{figure}


\begin{table}[h!]
\centering
\caption{Número de curvas ridges azul — Ambos os sexos}
\begin{tabular}{c c}
\hline
\textbf{Número de curvas ridges azul} & \textbf{Quantidade de indivíduos} \\ 
\hline
14 & 1  \\
15 & 1  \\
16 & 7  \\
17 & 13 \\
18 & 23 \\
19 & 38 \\
20 & 59 \\
21 & 95 \\
22 & 109 \\
23 & 105 \\
24 & 142 \\
25 & 108 \\
26 & 89 \\
27 & 71 \\
28 & 55 \\
29 & 36 \\
30 & 14 \\
31 & 10 \\
32 & 4  \\
33 & 3  \\
34 & 1  \\
35 & 2  \\
38 & 1  \\
\hline
\end{tabular}
\end{table}

\begin{figure}[h!]
\centering
\begin{tikzpicture}
\begin{axis}[
    ybar,
    bar width=7pt,
    xlabel={Número de curvas ridges azul},
    ylabel={Quantidade de indivíduos},
    xtick=data,
    enlarge x limits=0.05,
    ymin=0,
    width=15cm,
    height=9cm,
    grid=major,
]
\addplot coordinates {
    (14,1)
    (15,1)
    (16,7)
    (17,13)
    (18,23)
    (19,38)
    (20,59)
    (21,95)
    (22,109)
    (23,105)
    (24,142)
    (25,108)
    (26,89)
    (27,71)
    (28,55)
    (29,36)
    (30,14)
    (31,10)
    (32,4)
    (33,3)
    (34,1)
    (35,2)
    (38,1)
};
\end{axis}
\end{tikzpicture}
\caption{Número de curvas ridges azul — Ambos os sexos}
\end{figure}


\begin{table}[h!]
\centering
\caption{Número de curvas ridges vermelha — Homens}
\begin{tabular}{c c}
\hline
\textbf{Número de curvas ridges vermelha} & \textbf{Quantidade de indivíduos} \\ 
\hline
12 & 1  \\
13 & 7  \\
14 & 13 \\
15 & 16 \\
16 & 27 \\
17 & 51 \\
18 & 48 \\
19 & 77 \\
20 & 67 \\
21 & 62 \\
22 & 63 \\
23 & 49 \\
24 & 40 \\
25 & 30 \\
26 & 11 \\
27 & 14 \\
28 & 6  \\
29 & 1  \\
30 & 1  \\
31 & 2  \\
\hline
\end{tabular}
\end{table}

\begin{figure}[h!]
\centering
\begin{tikzpicture}
\begin{axis}[
    ybar,
    bar width=7pt,
    xlabel={Número de curvas ridges vermelha},
    ylabel={Quantidade de indivíduos},
    xtick=data,
    enlarge x limits=0.05,
    ymin=0,
    width=15cm,
    height=9cm,
    grid=major,
]
\addplot coordinates {
    (12,1)
    (13,7)
    (14,13)
    (15,16)
    (16,27)
    (17,51)
    (18,48)
    (19,77)
    (20,67)
    (21,62)
    (22,63)
    (23,49)
    (24,40)
    (25,30)
    (26,11)
    (27,14)
    (28,6)
    (29,1)
    (30,1)
    (31,2)
};
\end{axis}
\end{tikzpicture}
\caption{Número de curvas ridges vermelha — Homens}
\end{figure}


\begin{table}[h!]
\centering
\caption{Número de curvas ridges vermelha — Mulheres}
\begin{tabular}{c c}
\hline
\textbf{Número de curvas ridges vermelha} & \textbf{Quantidade de indivíduos} \\ 
\hline
10 & 1  \\
12 & 2  \\
13 & 4  \\
14 & 10 \\
15 & 9  \\
16 & 21 \\
17 & 32 \\
18 & 32 \\
19 & 41 \\
20 & 51 \\
21 & 48 \\
22 & 42 \\
23 & 35 \\
24 & 27 \\
25 & 21 \\
26 & 13 \\
27 & 8  \\
28 & 2  \\
30 & 1  \\
31 & 1  \\
\hline
\end{tabular}
\end{table}

\begin{figure}[h!]
\centering
\begin{tikzpicture}
\begin{axis}[
    ybar,
    bar width=7pt,
    xlabel={Número de curvas ridges vermelha},
    ylabel={Quantidade de indivíduos},
    xtick=data,
    enlarge x limits=0.05,
    ymin=0,
    width=15cm,
    height=9cm,
    grid=major,
]
\addplot coordinates {
    (10,1)
    (12,2)
    (13,4)
    (14,10)
    (15,9)
    (16,21)
    (17,32)
    (18,32)
    (19,41)
    (20,51)
    (21,48)
    (22,42)
    (23,35)
    (24,27)
    (25,21)
    (26,13)
    (27,8)
    (28,2)
    (30,1)
    (31,1)
};
\end{axis}
\end{tikzpicture}
\caption{Número de curvas ridges vermelha — Mulheres}
\end{figure}


\begin{table}[h!]
\centering
\caption{Número de curvas ridges vermelha — Ambos os sexos}
\begin{tabular}{c c}
\hline
\textbf{Número de curvas ridges vermelha} & \textbf{Quantidade de indivíduos} \\ 
\hline
10 & 1  \\
12 & 3  \\
13 & 11 \\
14 & 23 \\
15 & 25 \\
16 & 48 \\
17 & 83 \\
18 & 80 \\
19 & 118 \\
20 & 118 \\
21 & 110 \\
22 & 105 \\
23 & 84 \\
24 & 67 \\
25 & 51 \\
26 & 24 \\
27 & 22 \\
28 & 8  \\
29 & 1  \\
30 & 2  \\
31 & 3  \\
\hline
\end{tabular}
\end{table}

\begin{figure}[h!]
\centering
\begin{tikzpicture}
\begin{axis}[
    ybar,
    bar width=7pt,
    xlabel={Número de curvas ridges vermelha},
    ylabel={Quantidade de indivíduos},
    xtick=data,
    enlarge x limits=0.05,
    ymin=0,
    width=15cm,
    height=9cm,
    grid=major,
]
\addplot coordinates {
    (10,1)
    (12,3)
    (13,11)
    (14,23)
    (15,25)
    (16,48)
    (17,83)
    (18,80)
    (19,118)
    (20,118)
    (21,110)
    (22,105)
    (23,84)
    (24,67)
    (25,51)
    (26,24)
    (27,22)
    (28,8)
    (29,1)
    (30,2)
    (31,3)
};
\end{axis}
\end{tikzpicture}
\caption{Número de curvas ridges vermelha — Ambos os sexos}
\end{figure}


\begin{table}[h!]
\centering
\caption{Número de curvas ridges azul fechada — Homens}
\begin{tabular}{c c}
\hline
\textbf{Número de curvas ridges azul fechada} & \textbf{Quantidade de indivíduos} \\ 
\hline
4  & 1  \\
6  & 2  \\
7  & 7  \\
8  & 8  \\
9  & 27 \\
10 & 44 \\
11 & 41 \\
12 & 77 \\
13 & 74 \\
14 & 71 \\
15 & 79 \\
16 & 53 \\
17 & 36 \\
18 & 31 \\
19 & 17 \\
20 & 12 \\
21 & 2  \\
22 & 1  \\
23 & 1  \\
24 & 1  \\
25 & 1  \\
\hline
\end{tabular}
\end{table}

\begin{figure}[h!]
\centering
\begin{tikzpicture}
\begin{axis}[	
    ybar,
    width=15cm,
    height=9cm,
    xlabel={Número de curvas ridges azul fechada},
    ylabel={Quantidade de indivíduos},
    xtick=data,
    ymin=0,
    bar width=7pt,
    enlarge x limits=0.05,
    grid=major,
    %title={Número de curvas ridges azul fechada — Homens}
]
\addplot coordinates {
    (4,1)
    (6,2)
    (7,7)
    (8,8)
    (9,27)
    (10,44)
    (11,41)
    (12,77)
    (13,74)
    (14,71)
    (15,79)
    (16,53)
    (17,36)
    (18,31)
    (19,17)
    (20,12)
    (21,2)
    (22,1)
    (23,1)
    (24,1)
    (25,1)
};
\end{axis}
\end{tikzpicture}
\caption{Número de curvas ridges azul fechada — Homens}
\end{figure}


\begin{table}[h!]
\centering
\caption{Número de curvas ridges azul fechada — Mulheres}
\begin{tabular}{c c}
\hline
\textbf{Número de curvas ridges azul fechada} & \textbf{Quantidade de indivíduos} \\ 
\hline
6  & 1  \\
7  & 2  \\
8  & 5  \\
9  & 11 \\
10 & 23 \\
11 & 31 \\
12 & 42 \\
13 & 34 \\
14 & 55 \\
15 & 57 \\
16 & 51 \\
17 & 21 \\
18 & 21 \\
19 & 16 \\
20 & 10 \\
21 & 8  \\
22 & 6  \\
23 & 4  \\
24 & 1  \\
26 & 1  \\
29 & 1  \\
\hline
\end{tabular}
\end{table}

\begin{figure}[h!]
\centering
\begin{tikzpicture}
\begin{axis}[
    ybar,
    width=15cm,
    height=8cm,
    xlabel={Número de curvas ridges azul fechada},
    ylabel={Quantidade de indivíduos},
    ymin=0,
    bar width=8pt,
    enlarge x limits=0.02,
    xtick=data,
    grid=major,
    %title={Número de curvas ridges azul fechada — Mulheres}
]
\addplot coordinates {
    (6,1)
    (7,2)
    (8,5)
    (9,11)
    (10,23)
    (11,31)
    (12,42)
    (13,34)
    (14,55)
    (15,57)
    (16,51)
    (17,21)
    (18,21)
    (19,16)
    (20,10)
    (21,8)
    (22,6)
    (23,4)
    (24,1)
    (26,1)
    (29,1)
};
\end{axis}
\end{tikzpicture}
\caption{Número de curvas ridges azul fechada — Mulheres}
\end{figure}


\begin{table}[h!]
\centering
\caption{Número de curvas ridges azul fechada — Ambos os sexos}
\begin{tabular}{c c}
\hline
\textbf{Número de curvas ridges azul fechada} & \textbf{Quantidade de indivíduos} \\ 
\hline
4  & 1   \\
6  & 3   \\
7  & 9   \\
8  & 13  \\
9  & 38  \\
10 & 67  \\
11 & 72  \\
12 & 119 \\
13 & 108 \\
14 & 126 \\
15 & 136 \\
16 & 104 \\
17 & 57  \\
18 & 52  \\
19 & 33  \\
20 & 22  \\
21 & 10  \\
22 & 7   \\
23 & 5   \\
24 & 2   \\
25 & 1   \\
26 & 1   \\
29 & 1   \\
\hline
\end{tabular}
\end{table}

\begin{figure}[h!]
\centering
\begin{tikzpicture}
\begin{axis}[
    ybar,
    width=15cm,
    height=8cm,
    xlabel={Número de curvas ridges azul fechada},
    ylabel={Quantidade de indivíduos},
    ymin=0,
    bar width=8pt,
    enlarge x limits=0.02,
    xtick=data,
    grid=major,
    %title={Número de curvas ridges azul fechada — Ambos os sexos}
]
\addplot coordinates {
    (4,1)
    (6,3)
    (7,9)
    (8,13)
    (9,38)
    (10,67)
    (11,72)
    (12,119)
    (13,108)
    (14,126)
    (15,136)
    (16,104)
    (17,57)
    (18,52)
    (19,33)
    (20,22)
    (21,10)
    (22,7)
    (23,5)
    (24,2)
    (25,1)
    (26,1)
    (29,1)
};
\end{axis}
\end{tikzpicture}
\caption{Número de curvas ridges azul fechada — Ambos os sexos}
\end{figure}


\begin{table}[h!]
\centering
\caption{Número de curvas ridges azul aberta — Homens}
\begin{tabular}{c c}
\hline
\textbf{Número de curvas ridges azul aberta} & \textbf{Quantidade de indivíduos} \\ 
\hline
5  & 3   \\
6  & 12  \\
7  & 39  \\
8  & 80  \\
9  & 122 \\
10 & 131 \\
11 & 78  \\
12 & 73  \\
13 & 31  \\
14 & 11  \\
15 & 3   \\
16 & 3   \\
\hline
\end{tabular}
\end{table}

\begin{figure}[h!]
\centering
\begin{tikzpicture}
\begin{axis}[
    ybar,
    width=15cm,
    height=8cm,
    xlabel={Número de curvas ridges azul aberta},
    ylabel={Quantidade de indivíduos},
    ymin=0,
    bar width=10pt,
    enlarge x limits=0.02,
    xtick=data,
    xticklabels={5,6,7,8,9,10,11,12,13,14,15,16},
    %title={Número de curvas ridges azul aberta — Homens}
]
\addplot coordinates {
    (1,3)
    (2,12)
    (3,39)
    (4,80)
    (5,122)
    (6,131)
    (7,78)
    (8,73)
    (9,31)
    (10,11)
    (11,3)
    (12,3)
};
\end{axis}
\end{tikzpicture}
\caption{Número de curvas ridges azul aberta — Homens}
\end{figure}


\begin{table}[h!]
\centering
\caption{Número de curvas ridges azul aberta — Mulheres}
\begin{tabular}{c c}
\hline
\textbf{Número de curvas ridges azul aberta} & \textbf{Quantidade de indivíduos} \\ 
\hline
5  & 1   \\
6  & 18  \\
7  & 36  \\
8  & 69  \\
9  & 85  \\
10 & 71  \\
11 & 58  \\
12 & 27  \\
13 & 16  \\
14 & 14  \\
15 & 6   \\
\hline
\end{tabular}
\end{table}

\begin{figure}[h!]
\centering
\begin{tikzpicture}
\begin{axis}[
    ybar,
    width=15cm,
    height=8cm,
    xlabel={Número de curvas ridges azul aberta},
    ylabel={Quantidade de indivíduos},
    ymin=0,
    bar width=10pt,
    enlarge x limits=0.02,
    xtick=data,
    xticklabels={5,6,7,8,9,10,11,12,13,14,15},
    %title={Número de curvas ridges azul aberta — Mulheres}
]
\addplot coordinates {
    (1,1)
    (2,18)
    (3,36)
    (4,69)
    (5,85)
    (6,71)
    (7,58)
    (8,27)
    (9,16)
    (10,14)
    (11,6)
};
\end{axis}
\end{tikzpicture}
\caption{Número de curvas ridges azul aberta — Mulheres}
\end{figure}


\begin{table}[h!]
\centering
\caption{Número de curvas ridges azul aberta — Ambos os sexos}
\begin{tabular}{c c}
\hline
\textbf{Número de curvas ridges azul aberta} & \textbf{Quantidade de indivíduos} \\ 
\hline
5  & 4    \\
6  & 30   \\
7  & 75   \\
8  & 149  \\
9  & 207  \\
10 & 202  \\
11 & 136  \\
12 & 100  \\
13 & 47   \\
14 & 25   \\
15 & 9    \\
16 & 3    \\
\hline
\end{tabular}
\end{table}

\begin{figure}[h!]
\centering
\begin{tikzpicture}
\begin{axis}[
    ybar,
    width=15cm,
    height=8cm,
    xlabel={Número de curvas ridges azul aberta},
    ylabel={Quantidade de indivíduos},
    ymin=0,
    bar width=10pt,
    enlarge x limits=0.02,
    xtick=data,
    xticklabels={5,6,7,8,9,10,11,12,13,14,15,16},
    %title={Número de curvas ridges azul aberta — Ambos os sexos}
]
\addplot coordinates {
    (1,4)
    (2,30)
    (3,75)
    (4,149)
    (5,207)
    (6,202)
    (7,136)
    (8,100)
    (9,47)
    (10,25)
    (11,9)
    (12,3)
};
\end{axis}
\end{tikzpicture}
\caption{Número de curvas ridges azul aberta — Ambos os sexos}
\end{figure}


\begin{table}[h!]
\centering
\caption{Número de curvas ridges vermelha fechada — Homens}
\begin{tabular}{c c}
\hline
\textbf{Número de curvas ridges vermelha fechada} & \textbf{Quantidade de indivíduos} \\ 
\hline
3  & 3   \\
4  & 13  \\
5  & 16  \\
6  & 34  \\
7  & 47  \\
8  & 63  \\
9  & 81  \\
10 & 73  \\
11 & 76  \\
12 & 57  \\
13 & 48  \\
14 & 31  \\
15 & 21  \\
16 & 9   \\
17 & 10  \\
18 & 3   \\
19 & 1   \\
\hline
\end{tabular}
\end{table}

\begin{figure}[h!]
\centering
\begin{tikzpicture}
\begin{axis}[
    ybar,
    width=15cm,
    height=8cm,
    xlabel={Número de curvas ridges vermelha fechada},
    ylabel={Quantidade de indivíduos},
    ymin=0,
    bar width=10pt,
    enlarge x limits=0.02,
    xtick=data,
    xticklabels={3,4,5,6,7,8,9,10,11,12,13,14,15,16,17,18,19},
    %title={Número de curvas ridges vermelha fechada — Homens}
]
\addplot coordinates {
    (1,3)
    (2,13)
    (3,16)
    (4,34)
    (5,47)
    (6,63)
    (7,81)
    (8,73)
    (9,76)
    (10,57)
    (11,48)
    (12,31)
    (13,21)
    (14,9)
    (15,10)
    (16,3)
    (17,1)
};
\end{axis}
\end{tikzpicture}
\caption{Número de curvas ridges vermelha fechada — Homens}
\end{figure}


\begin{table}[h!]
\centering
\caption{Número de curvas ridges vermelha fechada — Mulheres}
\begin{tabular}{c c}
\hline
\textbf{Número de curvas ridges vermelha fechada} & \textbf{Quantidade de indivíduos} \\ 
\hline
3  & 1   \\
4  & 6   \\
5  & 8   \\
6  & 12  \\
7  & 30  \\
8  & 52  \\
9  & 47  \\
10 & 57  \\
11 & 58  \\
12 & 47  \\
13 & 25  \\
14 & 23  \\
15 & 15  \\
16 & 9   \\
17 & 5   \\
18 & 3   \\
19 & 1   \\
20 & 1   \\
21 & 1   \\
\hline
\end{tabular}
\end{table}

\begin{figure}[h!]
\centering
\begin{tikzpicture}
\begin{axis}[
    ybar,
    width=15cm,
    height=8cm,
    xlabel={Número de curvas ridges vermelha fechada},
    ylabel={Quantidade de indivíduos},
    ymin=0,
    bar width=10pt,
    enlarge x limits=0.02,
    xtick=data,
    xticklabels={3,4,5,6,7,8,9,10,11,12,13,14,15,16,17,18,19,20,21},
    %title={Número de curvas ridges vermelha fechada — Mulheres}
]
\addplot coordinates {
    (1,1)
    (2,6)
    (3,8)
    (4,12)
    (5,30)
    (6,52)
    (7,47)
    (8,57)
    (9,58)
    (10,47)
    (11,25)
    (12,23)
    (13,15)
    (14,9)
    (15,5)
    (16,3)
    (17,1)
    (18,1)
    (19,1)
};
\end{axis}
\end{tikzpicture}
\caption{Número de curvas ridges vermelha fechada — Mulheres}
\end{figure}


\begin{table}[h!]
\centering
\caption{Número de curvas ridges vermelha fechada — Ambos os sexos}
\begin{tabular}{c c}
\hline
\textbf{Número de curvas ridges vermelha fechada} & \textbf{Quantidade de indivíduos} \\ 
\hline
3  & 4    \\
4  & 19   \\
5  & 24   \\
6  & 46   \\
7  & 77   \\
8  & 115  \\
9  & 128  \\
10 & 130  \\
11 & 134  \\
12 & 104  \\
13 & 73   \\
14 & 54   \\
15 & 36   \\
16 & 18   \\
17 & 15   \\
18 & 6    \\
19 & 2    \\
20 & 1    \\
21 & 1    \\
\hline
\end{tabular}
\end{table}

\begin{figure}[h!]
\centering
\begin{tikzpicture}
\begin{axis}[
    ybar,
    width=15cm,
    height=8cm,
    xlabel={Número de curvas ridges vermelha fechada},
    ylabel={Quantidade de indivíduos},
    ymin=0,
    bar width=10pt,
    enlarge x limits=0.02,
    xtick=data,
    xticklabels={3,4,5,6,7,8,9,10,11,12,13,14,15,16,17,18,19,20,21},
    %title={Número de curvas ridges vermelha fechada — Ambos os sexos}
]
\addplot coordinates {
    (1,4)
    (2,19)
    (3,24)
    (4,46)
    (5,77)
    (6,115)
    (7,128)
    (8,130)
    (9,134)
    (10,104)
    (11,73)
    (12,54)
    (13,36)
    (14,18)
    (15,15)
    (16,6)
    (17,2)
    (18,1)
    (19,1)
};
\end{axis}
\end{tikzpicture}
\caption{Número de curvas ridges vermelha fechada — Ambos os sexos}
\end{figure}


\begin{table}[h!]
\centering
\caption{Número de curvas ridges vermelha aberta — Homens}
\begin{tabular}{c c}
\hline
\textbf{Número de curvas ridges vermelha aberta} & \textbf{Quantidade de indivíduos} \\ 
\hline
4  & 1   \\
5  & 4   \\
6  & 23  \\
7  & 39  \\
8  & 60  \\
9  & 94  \\
10 & 96  \\
11 & 84  \\
12 & 80  \\
13 & 56  \\
14 & 25  \\
15 & 15  \\
16 & 4   \\
17 & 5   \\
\hline
\end{tabular}
\end{table}

\begin{figure}[h!]
\centering
\begin{tikzpicture}
\begin{axis}[
    ybar,
    width=15cm,
    height=8cm,
    xlabel={Número de curvas ridges vermelha aberta},
    ylabel={Quantidade de indivíduos},
    ymin=0,
    bar width=10pt,
    enlarge x limits=0.02,
    xtick=data,
    xticklabels={4,5,6,7,8,9,10,11,12,13,14,15,16,17},
    %title={Número de curvas ridges vermelha aberta — Homens}
]
\addplot coordinates {
    (1,1)
    (2,4)
    (3,23)
    (4,39)
    (5,60)
    (6,94)
    (7,96)
    (8,84)
    (9,80)
    (10,56)
    (11,25)
    (12,15)
    (13,4)
    (14,5)
};
\end{axis}
\end{tikzpicture}
\caption{Número de curvas ridges vermelha aberta — Homens}
\end{figure}


\begin{table}[h!]
\centering
\caption{Número de curvas ridges vermelha aberta — Mulheres}
\begin{tabular}{c c}
\hline
\textbf{Número de curvas ridges vermelha aberta} & \textbf{Quantidade de indivíduos} \\ 
\hline
5  & 8   \\
6  & 18  \\
7  & 37  \\
8  & 46  \\
9  & 60  \\
10 & 56  \\
11 & 68  \\
12 & 50  \\
13 & 25  \\
14 & 19  \\
15 & 10  \\
16 & 1   \\
17 & 2   \\
18 & 1   \\
\hline
\end{tabular}
\end{table}


\begin{figure}[h!]
\centering
\begin{tikzpicture}
\begin{axis}[
    ybar,
    width=15cm,
    height=8cm,
    xlabel={Número de curvas ridges vermelha aberta},
    ylabel={Quantidade de indivíduos},
    ymin=0,
    bar width=10pt,
    enlarge x limits=0.02,
    xtick=data,
    xticklabels={5,6,7,8,9,10,11,12,13,14,15,16,17,18},
   	%title={Número de curvas ridges vermelha aberta — Mulheres}
]
\addplot coordinates {
    (1,8)
    (2,18)
    (3,37)
    (4,46)
    (5,60)
    (6,56)
    (7,68)
    (8,50)
    (9,25)
    (10,19)
    (11,10)
    (12,1)
    (13,2)
    (14,1)
};
\end{axis}
\end{tikzpicture}
\caption{Número de curvas ridges vermelha aberta — Mulheres}
\end{figure}

\begin{table}[h!]
\centering
\caption{Número de curvas ridges vermelha aberta — Ambos os sexos}
\begin{tabular}{c c}
\hline
\textbf{Número de curvas ridges vermelha aberta} & \textbf{Quantidade de indivíduos} \\ 
\hline
4  & 1    \\
5  & 12   \\
6  & 41   \\
7  & 76   \\
8  & 106  \\
9  & 154  \\
10 & 152  \\
11 & 152  \\
12 & 130  \\
13 & 81   \\
14 & 44   \\
15 & 25   \\
16 & 5    \\
17 & 7    \\
18 & 1    \\
\hline
\end{tabular}
\end{table}
\begin{figure}[h!]
\centering
\begin{tikzpicture}
\begin{axis}[
    ybar,
    width=15cm,
    height=8cm,
    xlabel={Número de curvas ridges vermelha aberta},
    ylabel={Quantidade de indivíduos},
    ymin=0,
    bar width=10pt,
    enlarge x limits=0.02,
    xtick=data,
    xticklabels={4,5,6,7,8,9,10,11,12,13,14,15,16,17,18},
    %title={Número de curvas ridges vermelha aberta — Ambos os sexos}
]
\addplot coordinates {
    (1,1)
    (2,12)
    (3,41)
    (4,76)
    (5,106)
    (6,154)
    (7,152)
    (8,152)
    (9,130)
    (10,81)
    (11,44)
    (12,25)
    (13,5)
    (14,7)
    (15,1)
};
\end{axis}
\end{tikzpicture}
\caption{Número de curvas ridges vermelha aberta — Ambos os sexos}
\end{figure}





\end{document}