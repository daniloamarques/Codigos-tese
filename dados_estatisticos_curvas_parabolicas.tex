\documentclass[a4paper,12pt]{article}

% --- Pacotes essenciais ---
\usepackage[utf8]{inputenc}   % Codificação de caracteres
\usepackage[T1]{fontenc}      % Acentos corretos
\usepackage[brazil]{babel}    % Idioma português do Brasil
\usepackage{booktabs}         % Linhas mais elegantes nas tabelas
\usepackage{caption}          % Melhor controle sobre legendas
\usepackage{geometry}         % Controle de margens
\usepackage{array}            % Controle de largura das colunas
\usepackage{multirow}         % Combinar células verticalmente
\usepackage{float}            % Permite usar [H] para fixar tabelas
\usepackage{siunitx}          % Alinhamento numérico e formatação de números
\usepackage{pgfplots}
\pgfplotsset{compat=1.18}
% --- Configurações de layout ---
\geometry{a4paper, margin=2.5cm}
\captionsetup{justification=centering, font=small}

% --- Formatação de números decimais opcional, vírgula como separador ---
\sisetup{
  output-decimal-marker = {,}
}

\begin{document}

\begin{table}[h!]
\centering
\caption{Número de curvas parabólicas em Homens}
\begin{tabular}{c c}
\hline
\textbf{Número de curvas parabólicas} & \textbf{Quantidade de indivíduos} \\ 
\hline
17 & 1   \\
18 & 0   \\
19 & 4   \\
20 & 8   \\
21 & 14  \\
22 & 17  \\
23 & 39  \\
24 & 78  \\
25 & 136 \\
26 & 139 \\
27 & 105 \\
28 & 40  \\
29 & 5   \\
\hline
\end{tabular}
\end{table}

\begin{figure}[h!]
\centering
\begin{tikzpicture}
\begin{axis}[
    ybar,
    bar width=8pt,
    xlabel={Número de curvas parabólicas},
    ylabel={Quantidade de indivíduos},
    xtick=data,
    enlarge x limits=0.05,
    ymin=0,
    width=14cm,
    height=8cm,
    grid=major,
]
\addplot coordinates {
    (17,1)
    (18,0)
    (19,4)
    (20,8)
    (21,14)
    (22,17)
    (23,39)
    (24,78)
    (25,136)
    (26,139)
    (27,105)
    (28,40)
    (29,5)
};
\end{axis}
\end{tikzpicture}
\caption{Número de curvas parabólicas em Homens}
\end{figure}

\begin{table}[h!]
\centering
\caption{Número de curvas parabólicas em Mulheres}
\begin{tabular}{c c}
\hline
\textbf{Número de curvas parabólicas} & \textbf{Quantidade de indivíduos} \\ 
\hline
15 & 1  \\
16 & 1  \\
17 & 3  \\
18 & 6  \\
19 & 9  \\
20 & 10 \\
21 & 14 \\
22 & 25 \\
23 & 22 \\
24 & 58 \\
25 & 77 \\
26 & 90 \\
27 & 59 \\
28 & 20 \\
29 & 5  \\
30 & 1  \\
\hline
\end{tabular}
\end{table}

\begin{figure}[h!]
\centering
\begin{tikzpicture}
\begin{axis}[
    ybar,
    bar width=8pt,
    xlabel={Número de curvas parabólicas},
    ylabel={Quantidade de indivíduos},
    xtick=data,
    enlarge x limits=0.05,
    ymin=0,
    width=14cm,
    height=8cm,
    grid=major,
]
\addplot coordinates {
    (15,1)
    (16,1)
    (17,3)
    (18,6)
    (19,9)
    (20,10)
    (21,14)
    (22,25)
    (23,22)
    (24,58)
    (25,77)
    (26,90)
    (27,59)
    (28,20)
    (29,5)
    (30,1)
};
\end{axis}
\end{tikzpicture}
\caption{Número de curvas parabólicas em Mulheres}
\end{figure}


\begin{table}[h!]
\centering
\caption{Número de curvas parabólicas — Ambos os sexos}
\begin{tabular}{c c}
\hline
\textbf{Número de curvas parabólicas} & \textbf{Quantidade de indivíduos} \\ 
\hline
15 & 1   \\
16 & 1   \\
17 & 4   \\
18 & 6   \\
19 & 13  \\
20 & 18  \\
21 & 28  \\
22 & 42  \\
23 & 61  \\
24 & 136 \\
25 & 213 \\
26 & 229 \\
27 & 164 \\
28 & 60  \\
29 & 10  \\
30 & 1   \\
\hline
\end{tabular}
\end{table}


\begin{figure}[h!]
\centering
\begin{tikzpicture}
\begin{axis}[
    ybar,
    bar width=8pt,
    xlabel={Número de curvas parabólicas},
    ylabel={Quantidade de indivíduos},
    xtick=data,
    enlarge x limits=0.05,
    ymin=0,
    width=14cm,
    height=8cm,
    grid=major,
]
\addplot coordinates {
    (15,1)
    (16,1)
    (17,4)
    (18,6)
    (19,13)
    (20,18)
    (21,28)
    (22,42)
    (23,61)
    (24,136)
    (25,213)
    (26,229)
    (27,164)
    (28,60)
    (29,10)
    (30,1)
};
\end{axis}
\end{tikzpicture}
\caption{Número de curvas parabólicas — Ambos os sexos}
\end{figure}



\end{document}